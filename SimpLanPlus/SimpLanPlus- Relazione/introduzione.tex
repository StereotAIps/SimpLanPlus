\section{Introduzione a SimpLanPlus}
In questo progetto viene sviluppato un compilatore in grado di eseguire SimpLanPlus, un linguaggio imperativo che estende SimpLan \footnote{https://virtuale.unibo.it/pluginfile.php/1550508/mod\_resource/content/1/1.SimpLan\_AND\_ANTLR.pdf}, le sue caratteristiche sono elencate come segue:
\begin{itemize}
\item I tipi ammessi sono: interi, booleani o void.
\item Le dichiarazioni di variabili sono della forma  \textit{type ID} e non ammettono inizializzazione in fase di dichiarazione.
\item Il corpo di una funzione può contenere dichiarazioni e statement, e facoltativamente, possono restituire il valore ottenuto da un'espressione. La dichiarazioni di variabli con lo stesso id è possibile Nel corpo di una funzione è possibile accedere a variabili globali e chiamare altre funzioni, se precedentemente dichiarate. Inoltre, è possibile dichiarare una funzione all'interno del corpo di una funzione (se questa non hanno lo stesso id) ed è possibile avere funzioni ricorsive, ma non mutuamente ricorsive.
\end{itemize}

Il codice utilizzato per lo sviluppo del progetto è disponibile nella repository GitHub nel link a piè di pagina \footnote{}.

\subsection{Grammatica di SimpLanPlus}
In questa sezione vengono evidenziate le differenze e i cambiamenti effettuati tra la grammatica fornita inizialmente e quella effettivamente utilizzata in questo progetto con lo scopo di raggiungere gli obiettivi e le specifiche preposte. Segue grammatica SimpLanPlus.

Al fine di rispettare i requisiti del linguaggio, la grammatica è stata modificata introducendo gli elementi \textit{stms: (stm)+;} e \textit{stme: (stm)* exp;} in modo tale da differenziare i contenuti dei blocchi then e else in entrambi i tipi di if. Più nel dettaglio, l'if presente nell'elemento \textit{stm} diventa \textit{'if' '(' exp ')' '{' left=stms '}' ('else' '{' right=stms '}')?} e l'if presente nell'elemento \textit{exp} diventa \textit{'if' '(' cond=exp ')' '{' left=stme '}' 'else' '{' right=stme '}'}.

Nelle produzioni dell'elemento \textit{exp} sono state aggiunte delle "etichette" che permettono di semplificare i controlli sugli elementi presenti nella produzione. Tali etichette sono \textit{leftExp}, \textit{op} e \textit{rightExp}. Le stesse notazioni sono usate negli elementi riguardanti gli if condizionali. 

Inoltre, per avere maggiore controllo su possibili errori sintattici, è necessario aggiungere in coda al programma in input la notazione \textit{EOF} (end of file).

Per finire, sono state aggiunte delle notazioni sotto forma di \textit{\#notazione} per ogni produzione di ogni elemento che contiene più produzioni per semplificare la gestione di tali produzioni e creare le funzioni necessarie nel visitor. 

%\noindent\fbox{%
%    \parbox{\textwidth}{%
%\lstinputlisting[language=java]{./Code/SimpLanPlus.java}
%}%
%}

\begin{minted}[xleftmargin=20pt,linenos]{antlr}
grammar SimpLanPlus ;
prog   : exp EOF                     #expProg
       | (dec)+ (stm)* (exp)?  EOF   #letProg
       ;
dec    : type ID ';'              #varDec
       | type ID '(' ( param ( ',' param)* )? ')' '{' body '}' #funDec
       ;
param  : type ID
       ;
body   : (dec)* (stm)* (exp)?
	   ;
type   : 'int'
       | 'bool'
       | 'void'
       ;
stm    : ID '=' exp ';' #asgStm                                                                  
       | ID '(' (exp (',' exp)* )? ')' ';'#callStm
       | 'if' '(' exp ')' '{' left=stms '}' ('else' '{' right=stms '}')? #ifStm
	   ;
stms   : (stm)+
       ;
stme   : (stm)* exp
       ;
exp    :  INTEGER #intExp
       | ('true' | 'false') #boolExp
       | ID #idExp
       | '!' exp #notExp
       | leftExp=exp (op='*' | op='/') rightExp=exp #numExp
       | leftExp=exp (op='+' | op='-') rightExp=exp #numExp
       | leftExp=exp  '==' rightExp=exp #eqExp
       | leftExp=exp (op='>' | op='<' | op='>=' | op='<=' ) rightExp=exp #compExp
       | leftExp=exp (op='&&' | op='||') rightExp=exp #opExp
       | 'if' '(' cond=exp ')' '{' left=stme '}' 'else' '{' right=stme '}' #ifExp
       | '(' exp ')' #baseExp
       | ID '(' (exp (',' exp)* )? ')' #callExp
       ;

\end{minted}

\subsection{Struttura compilatore}
Lo schema utilizzato per la stuttura del compilatore è stata costruita su modello della struttura utilizzata per SimpLan, dove: 
\begin{itemize}
    \item \textbf{ast}: contiene le implementazioni dei nodi dell'albero sintattico e dei visitor 
    \item \textbf{evaluator}: contiene le implementazioni per eseguire il codice assembly generato
    \item \textbf{mainPackage}: contiene il main del compilatore, il file del codice in input e del codice assembly generato
    \item \textbf{parser}: contiene le classi generate da ANTLR sulla base della grammatica SimpLanPlus. 
    \item \textbf{svm}: contiene le classi generate da ANTLR sulla base della grammatica SVM.
    \item \textbf{symboltable}: contiene le classi che servono a gestire la symbol table.
    \item \textbf{utils}: contiene le classi utilizzate come supporto alla gestione delle strutture utilizzate.
    
\end{itemize}

\subsection{Software e tools}
Il codice di questo elaborato è stato implementato usando come IDE IntelliJ IDEA \footnote{https://www.jetbrains.com/idea/} sviluppato da JetBrains, un ambiente di sviluppo integrato scritto in Java per lo sviluppo di software scritto in Java. Per quanto riguarda il versioning, è stato utilizzato GitHub, Inc. \footnote{https://github.com} sviluppato da Microsoft Corporation, un servizio di hosting per lo sviluppo di software e il controllo del versioning tramite Git. 
Per la generazione del parser, è stato utilizzato il software ANTLR \footnote{https://www.antlr.org} (ANother Tool for Language Recognition), un generatore di parser che utilizza un algoritmo per il parsing LL.

La versione di Java utilizzata come SDK è la \textit{11.0.18} scaricabile al link \footnote{https://www.oracle.com/java/technologies/javase/11-0-18-relnotes.html} a fondo pagina.
L'unica libreria importata è la seguente \textit{antlr-4.12.0-complete.jar}, anche questa scaricabile gratuitamente tramite il link \footnote{https://www.antlr.org/download.html} a fondo pagina. 


