\section{Esercizio 4}
Lo scopo di questo esercizio è quello di estendere l'interprete di SimpLan per implementare l'interprete di SimpLanPlus. 

\subsection{Interprete SVM}
Per perseguire gli obiettivi di questa fase, è stato importato al progetto il file \textit{SVM.g4}, contentente la grammatica dell'interprete di base, nel package \textit{svm}. In seguito è stato utilizzato lo stesso metodo già utlizzato per la grammatica del compilatore \textit{SimpLanPlus.g4}, ovvero è stato utilizzato ANTLR per generare il codice del visitor della SVM. Successivamente, sono stati creati i metodi per la generazione di codice in ogni nodo dell'albero creato precedentementemente per costruire l'AST. 

Nel main sono state aggiunte le componenti che scrivono in un file di appoggio \textit{prova.simplan.asm} (contenuto nel package \textit{mainPackage}) il codice generato dal programma. Venogno poi utilizzati gli stessi step usati precedentemente per leggere il nuovo file creato, estrarre i token ed effettuare il parser. In fine, viene eseguita la VM per eseguire il codice generato.

\subsection{Esercizi}
In questa sezione testiamo i seguenti 4 codici forniti nelle specifiche del progetto.

\begin{enumerate}
\item \textbf{Codice 1}
\begin{minted}[xleftmargin=20pt,linenos]{java}
int x ;
void f(int n){
	if (n == 0) { n = 0 ; }       // n e` gia` uguale a 0; equivale a fare skip
	else { x = x * n ; f(n-1) ; }
}
x = 1 ;
f(10)
\end{minted}
Il codice viene compilato ed eseguito correttamente ed il risultato ottenuto è 0 dal momento in cui la funzione chiamata è di tipo void, ovvero non restituisce nessun valore. Il valore di ritorno è salvato nel registro a0 che viene inzialmente settato a 0.

\begin{minted}[xleftmargin=20pt,linenos]{java}
Parsing in progress....
Parse completed without errors!
Checking semantic errors...
Visualizing AST...
Checking type errors...
Type checking ok! Type of the program is: [T]Void
Code generated! Assembling and running generated code.
Starting Virtual Machine...
*** STACK ***
Code executed with success!
The result is: 0
\end{minted}

\item \textbf{Codice 2}
\begin{minted}[xleftmargin=20pt,linenos]{java}
int u ;
int f(int n){
        int y ;
	y = 1 ;
	if (n == 0) { y }
	else { y = f(n-1) ; y*n }
}
u = 6 ;
f(u)
\end{minted}
Il codice viene compilato ed eseguito correttamente ed il risultato ottenuto è 720.

\begin{minted}[xleftmargin=20pt,linenos]{java}
Parsing in progress....
Parse completed without errors!
Checking semantic errors...
Visualizing AST...
Checking type errors...
Type checking ok! Type of the program is: [T]Void
Code generated! Assembling and running generated code.
Starting Virtual Machine...
*** STACK ***
Code executed with success!
The result is: 720
\end{minted}

\item \textbf{Codice 3}
\begin{minted}[xleftmargin=20pt,linenos]{java}
int u ;
void f(int m, int n){
	if (m>n) { u = m+n ;}
	else { int x ; x = 1 ; f(m+1,n+1) ; }
}
f(5,4) ;
\end{minted}
Il compilatore si interrompe al controllo degli errori presenti nel parser nel momento in cui il codice non è coerente con la gammatica. Infatti, nel corpo del ramo else è presente una dichiarazione di variabile. Le dichiarazioni non sono consentite nei branch del costrutto if.

\begin{minted}[xleftmargin=20pt,linenos]{java}
Parsing in progress....
[!] An error occurred at line 4, character 8 :extraneous input 'int' 
expecting {'if', ID}
[!] An error occurred at line 4, character 14 :no viable alternative 
at input 'x;'
parser.ParserErrorHandler@15d0c81b
\end{minted}
\end{enumerate}